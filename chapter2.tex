\chapter{Monopolarity}
\label{cap:mono}

\section{Previous results}
\label{sec:previous}

\begin{proposition}
    Union of two monopolar graphs is monopolar.
\end{proposition}

\begin{corollary}
    Every minimal obstruction for monopolarity is a connected graph.
\end{corollary}

%TODO: Decidir si incluir los resultados de cordales y cograficas como
%preliminar (enunciar los resultados)

\section{Chordal \texorpdfstring{$(P_5,\text{house})$}{(P5,house)}-free graphs}
\label{sec:chordal}

\begin{figure}[h]
    \centering
    \begin{tikzpicture}[every node/.style={vertex, fill=black},
        node distance=1cm, on grid]
        \begin{scope}[xshift=0cm]
            \node (a) {};
            \node (b) [right= of a] {};
            \node (c) [below= of a] {};
            \node (d) [right= of c] {};
            \node (e) [below= of c] {};
            \node (f) [below= of d] {};
            \draw[edge] (a) to (b);
            \draw[edge] (a) to (c);
            \draw[edge] (a) to (d);
            \draw[edge] (b) to (c);
            \draw[edge] (b) to (d);
            \draw[edge] (c) to (d);
            \draw[edge] (c) to (e);
            \draw[edge] (c) to (f);
            \draw[edge] (d) to (e);
            \draw[edge] (d) to (f);
            \draw[edge] (e) to (f);
        \end{scope}

        \begin{scope}[xshift=-3cm]
            \node (a) {};
            \node (b) [right= of a] {};
            \node (c) [below= of a] {};
            \node (d) [right= of c] {};
            \node (e) [below= of c] {};
            \node (f) [below= of d] {};
            \draw[edge] (a) to (b);
            \draw[edge] (a) to (c);
            \draw[edge] (a) to (d);
            \draw[edge] (b) to (c);
            \draw[edge] (b) to (d);
            \draw[edge] (c) to (d);
            \draw[edge] (c) to (e);
            \draw[edge] (c) to (f);
            \draw[edge] (d) to (f);
            \draw[edge] (e) to (f);
        \end{scope}

        \begin{scope}[xshift=-6cm]
            \node (a) {};
            \node (b) [right= of a] {};
            \node (c) [below= of a] {};
            \node (d) [right= of c] {};
            \node (e) [below= of c] {};
            \node (f) [below= of d] {};
            \draw[edge] (a) to (b);
            \draw[edge] (a) to (c);
            \draw[edge] (a) to (d);
            \draw[edge] (b) to (d);
            \draw[edge] (c) to (d);
            \draw[edge] (c) to (e);
            \draw[edge] (c) to (f);
            \draw[edge] (d) to (f);
            \draw[edge] (e) to (f);
        \end{scope}

        \begin{scope}[xshift=3cm]
            \node (a) {};
            \node (b) [below right= of a] {};
            \node (c) [below left= of a] {};
            \node (d) [below left= of b] {};
            \node (e) [rotate around={150:(d)}] {};
            \node (f) [rotate around={210:(d)}] {};
            \draw[edge] (a) to (b);
            \draw[edge] (a) to (c);
            \draw[edge] (a) to (d);
            \draw[edge] (b) to (d);
            \draw[edge] (c) to (d);
            \draw[edge] (d) to (f);
            \draw[edge] (d) to (e);
            \draw[edge] (e) to (f);
        \end{scope}

        \begin{scope}[xshift=-4cm, yshift=-4cm]
            \node (a) at (-1,0) {};
            \node (b) [right=2 of a] {};
            \node (c) [rotate around={105:(a)}] {};
            \node (d) [rotate around={165:(a)}] {};
            \node (e) [rotate around={-105:(a)}] {};
            \node (f) [rotate around={-165:(a)}] {};
            \node (g) [rotate around={-105:(b)}] {};
            \node (h) [rotate around={-165:(b)}] {};
            \node (i) [rotate around={105:(b)}] {};
            \node (j) [rotate around={165:(b)}] {};
            \draw[edge] (a) to (b);
            \draw[edge] (a) to (c);
            \draw[edge] (a) to (d);
            \draw[edge] (a) to (e);
            \draw[edge] (a) to (f);
            \draw[edge] (c) to (d);
            \draw[edge] (e) to (f);
            \draw[edge] (b) to (g);
            \draw[edge] (b) to (h);
            \draw[edge] (b) to (i);
            \draw[edge] (b) to (j);
            \draw[edge] (g) to (h);
            \draw[edge] (i) to (j);
        \end{scope}

        \begin{scope}[xshift=0cm, yshift=-4.5cm]
            \node (a) {};
            \node (b) [above right= of a] {};
            \node (c) [above left= of a] {};
            \node (d) [rotate around={-105:(b)}] {};
            \node (e) [rotate around={-165:(b)}] {};
            \node (f) [rotate around={105:(c)}] {};
            \node (g) [rotate around={165:(c)}] {};
            \draw[edge] (a) to (b);
            \draw[edge] (a) to (c);
            \draw[edge] (b) to (c);
            \draw[edge] (b) to (d);
            \draw[edge] (b) to (e);
            \draw[edge] (d) to (e);
            \draw[edge] (c) to (f);
            \draw[edge] (c) to (g);
            \draw[edge] (f) to (g);
        \end{scope}

        \begin{scope}[xshift=4cm, yshift=-4cm]
            \node (a) at (-1,0) {};
            \node (b) [right=2 of a] {};
            \node (c) [rotate around={120:(a)}] {};
            \node (d) [rotate around={180:(a)}] {};
            \node (e) [rotate around={240:(a)}] {};
            \node (f) [rotate around={-120:(b)}] {};
            \node (g) [rotate around={-180:(b)}] {};
            \node (h) [rotate around={-240:(b)}] {};
            \draw[edge] (a) to (b);
            \draw[edge] (a) to (c);
            \draw[edge] (a) to (d);
            \draw[edge] (a) to (e);
            \draw[edge] (c) to (d);
            \draw[edge] (d) to (e);
            \draw[edge] (b) to (f);
            \draw[edge] (b) to (g);
            \draw[edge] (b) to (h);
            \draw[edge] (f) to (g);
            \draw[edge] (g) to (h);
        \end{scope}
    \end{tikzpicture}
    \caption{The minimal obstructions for monopolarity in chordal
    $P_5$-free graphs.}
    \label{fig:chordal-p5-free-obstructions}
\end{figure}

\begin{proposition}
    A chordal $(P_5,\text{house})$-free graph is monopolar if and only if it
    does not contain any of the graphs depicted in
    \cref{fig:chordal-p5-free-obstructions} as an induced subgraph.
\end{proposition}

\section{\texorpdfstring{$(P_5,\text{house})$}{(P5,house)}-free graphs with \texorpdfstring{$C_5$}{C5}}
\label{sec:c5}

\begin{proposition}
    Up to symmetry, there is only one monopolar partition of $C_5$.
\end{proposition}

\begin{proposition}
    Let $G$ be a $(P_5,\text{house})$-free graph containing an induced $C_5$. If
    $G$ is a minimal obstruction for monopolarity, then every vertex must be
    adjacent to at least one vertex of the $C_5$.
\end{proposition}

\begin{proposition}
    Let $G$ be a $(P_5,\text{house})$-free monopolar graph with an induced
    $C_5$. Let $C$ be a $C_5$ in $G$. Then:
    \begin{enumerate}[(i)]
        \item For every pair of consective vertices $x$ and $y$ in $C$,
        $N_{xy}=\varnothing$.
        \item For every vertex $x$ in $C$, $N_x = \varnothing$.
        \item For every four vertices $w$, $x$, $y$ and $z$ in $C$, $N_{wxyz}=\varnothing$.
        \item For every triple of vertices non-consecutive vertices $x$, $y$ and $z$ in $C$,
        $N_{xyz}=\varnothing$.
    \end{enumerate}
\end{proposition}

\begin{figure}[h]
    \centering
    \begin{tikzpicture}[every node/.style={vertex, fill=black},
        node distance=1cm, on grid]
        \begin{scope}[xshift=0cm]
            \foreach \i in {0,...,4}
                \node (\i)  at ({(360/5)*\i + 90}:1){};

            \node (5) at (0,0) {};
            
            \foreach \i in {0,...,4}{
                \draw [edge] let \n1={int(mod(\i+1,5))} in (\i) to (\n1);
                \draw [edge] (\i) to (5);
            }
        \end{scope}
    \end{tikzpicture}
    
    \caption{The minimal obstructions for monopolarity in
    $(P_5,\text{house})$-free graphs containing an induced $C_5$.}
    \label{fig:p5-house-obstruction-with-c5}
\end{figure}

\begin{proposition}
    A $(P_5,\text{house})$-free graph $G$ with and induced $C_5$ is monopolar if
    and only if it does not contain any of the graphs depicted in
    \cref{fig:p5-house-obstruction-with-c5} as an induced subgraph.
\end{proposition}

% TO DO: Structure of monopolar partitions given a C5
% TO DO: Lemma of minimal obstructions containing a C5
% TO DO: Algorithm for monopolarity in C5-containing graphs

\section{\texorpdfstring{$(P_5,\text{house})$}{(P5,house)}-free graphs without
\texorpdfstring{$C_5$}{C5}}
\label{sec:no-c5}

